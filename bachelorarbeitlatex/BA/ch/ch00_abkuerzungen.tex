\chapter*{Abkürzungs- und Symbolverzeichnis}\markboth{Abkürzungs- und Symbolverzeichnis}{Abkürzungs- und Symbolverzeichnis}
\addcontentsline{toc}{chapter}{\textbf{Abkürzungs- und Symbolverzeichnis}}
\renewcommand{\arraystretch}{1.5}
%

\vspace*{-1cm}
\section*{Griechische Symbole}
\begin{table}[ht!]
    \centering
\begin{tabular}{m{0.15\textwidth}m{0.75\textwidth}}
        \rowcolor{white}
Symbol   & Bedeutung          \\
        \midrule
$\alpha$ & Hilfswert zur Berechnung des Zielpunktabstandes     \\
$\HeatTransferCoefficientAbsorberBack$ & Wärmeübergangskoeffizient der Rückseite der Absorberwabe     \\
$\HeatTransferCoefficientAbsorberFront$ & Wärmeübergangskoeffizient der Vorderseite der Absorberwabe      \\
$\alpha_{\mathrm{i\to r}}$ & Wärmeübergangskoeffizient der Transportzone     \\
$\alpha_{\mathrm{sol}}$ & Absorptionskoeffizient des Receivers    \\
$\beta$ & Lichteinfallswinkel auf die Heliostaten \\
$\epsilon$ & Emissionskoeffizient der Absorberoberfläche \\
$\kappa$ & Streuungsfaktor der Zielpunktstrategie \\
$\lambda_{\mathrm{cer}}$ & Wärmeleitfähigkeit der Keramik \\
$\lambda_{\mathrm{comb}}$ & Wärmeleitfähigkeit der Absorberwabe  \\
$\lambda_{\mathrm{ins}}$ & Wärmeleitfähigkeit der Isolierung   \\
$\lambda_{\mathrm{pipe}}$ & Wärmeleitfähigkeit des Rohrs   \\
$\xi$ & Slack Variable für soft constraints   \\
$\xi_{\mathrm{rad}}$ & Gewichtungsfaktor zur Verteilung der Einstrahlungsleistung  \\
$\sigma$ & Boltzmann-Konstante   \\
$\tau$ & Diskrete Zeitpunkte auf den finiten Elementen der Kollokation   \\
$\phi$ & Strahlungsfluss   \\
    \end{tabular}
\end{table}
\clearpage
\newpage \vspace*{-1cm}

\renewcommand{\arraystretch}{0.93}
\section*{Lateinische Symbole}
\begin{longtable}{p{0.15\textwidth}p{0.75\textwidth}}
    Symbol                           & Bedeutung                                                        \\
    \hline
    \endfirsthead
    Symbol                           & Bedeutung                                                        \\
    \hline
    \endhead
    \endfoot
    \endlastfoot
    $A$                              & Fläche des Receivers                                             \\
    $A_{\mathrm{abs}}$               & Oberfläche des Absorbers                                         \\
    $A_{\mathrm{comb,back}}$         & Freiliegende Oberfläche der Rückseite der Absorberwabe           \\
    $A_{\mathrm{comb,front}}$        & Freiliegende Oberfläche der Vorderseite der Absorberwabe         \\
    $A_{\mathrm{i\to r}}$            & Oberfläche der Transportzone                                     \\
    $A_{\mathrm{solid}}$             & Oberfläche des Absorbers abzüglich der Luftschlitze              \\
    $c_{\mathrm{abs}}$               & Wärmekapazität des Absorbers                                     \\
    $D$                              & Dämpfungsfaktor eines PT2-Gliedes                                \\
    $d_{\mathrm{plate}}$             & Blendendurchmesser                                               \\
    $\boldsymbol{e}$                 & Regelfehler                                                      \\
    $F$                              & Flussdichte                                                      \\
    $\SpecificEnthalpyTAmbient$      & Spezifische Enthalpie der Umgebungsluft                          \\
    $\EnthalpyFlowTAmbient$          & Enthalpiestrom der Umgebungsluft                                 \\
    $\SpecificEnthalpyTInlet{x}$     & Spezifische Enthalpie der Luft an der Stelle x                   \\
    $\EnthalpyFlowTInlet{x}$         & Enthalpiestrom der Luft an der Stelle x                          \\
    $h_{\mathrm{out}}$               & Spezifische Enthalpie der Austrittsluft                          \\
    $\dot{H}_{\mathrm{out}}$         & Enthalpiestrom der Austrittsluft                                 \\
    $\SpecificEnthalpyTReturn{x}$    & Spezifische Enthalpie der Rückführluft an der Stelle x           \\
    $\EnthalpyFlowTReturn{x}$        & Enthalpiestrom der Rückführluft an der Stelle x                  \\
    $\dot{H}_{\mathrm{sector}}$      & Enthalpiestrom eines Sektors                                     \\
    $\boldsymbol{J}$                 & Kostenfunktion                                                   \\
    $K_p$                            & Proportionalitätsfaktor eines PT2-Gliedes                        \\
    $l_{\mathrm{comb}}$              & Tiefe der Absorberwabe                                           \\
    $\MDotAbsorberCup$               & Luftmassenstrom durch einen Absorbercup                          \\
    $m_{\mathrm{abs}}$               & Masse der Absorberwabe                                           \\
    $\MDotReceiver$                  & Luftmassenstrom durch den gesamten Receiver                      \\
    $N_1$                            & Oberer Regelungshorizont                                         \\
    $N_2$                            & Prädiktionshorizont                                              \\
    NE                               & Anzahl der finiten Elemente für die Kollokation                  \\
    $N_u$                            & Unterer Regelungshorizont                                        \\
    $\boldsymbol{p}$                 & Modellparameter                                                  \\
    $P$                              & Hilfsgröße zur Bestimmung des Wärmeverlustes eines Headers       \\
    $P_{\mathrm{sol}}$               & Absorbierte solare Einstrahlungsleistung                         \\
    $\boldsymbol{p}_{tv}$            & Zeitveränderliche Modellparameter                                \\
    $Q$                              & Strahlungsenergie                                                \\
    $\QDotAbsorberCombBack$          & Wärmestrom in die Luft an der Vorderseite der Absorberwabe       \\
    $\QDotAbsorberCombFront$         & Wärmestrom in die Luft an der Rückseite der Absorberwabe         \\
    $\QDotConduction$                & Wärmeleitung von der Vorder- zur Rückseite der Absorberwabe      \\
    $\QDotLossAbsorberCupInternal$   & Wärmeverlust der Transportzone                                   \\
    $\QDotLossAbsorberCupConvective$ & Wärmeverlust der Absorberwabe durch Konvektion                   \\
    $\QDotLoss{header}$              & Wärmeverlust eines Headers                                       \\
    $\QDotLossAbsorberCupRadiative$  & Wärmeverlust durch Strahlung                                     \\
    $\QDotLoss{tub}$                 & Wärmeverlust des Absorberkelches                                 \\
    $\QDotLoss{tube,1}$              & Wärmeverlust des Rohrstücks inklusive Endstück des Kelches       \\
    $\QDotLoss{tube,2}$              & Wärmeverlust des Rohrstücks exklusive Endstück des Kelches       \\
    $\QDotSol$                       & Wärmestrom durch die solare Einstrahlung                         \\
    $\boldsymbol{r}$                 & Referenztrajektorie                                              \\
    $r$                              & Abstand zweier Zielpunkte                                        \\
    $t$                              & Zeit                                                             \\
    $T$                              & Zeitkonstante eines PT2-Gliedes                                  \\
    $\TAbsorberBack$                 & Temperatur der Vorderseite der Absorberwabe                      \\
    $\TAbsorberFront$                & Temperatur der Rückseite der Absorberwabe                        \\
    $\TAmbient$                      & Temperatur der Umgebungsluft                                     \\
    $\TInlet{x}$                     & Temperatur der Luft an der Stelle x                              \\
    $T_{\mathrm{m,back}}$            & Mittlere Temperatur zwischen Mitte und Rückseite des Absorbers   \\
    $T_{\mathrm{m,front}}$           & Mittlere Temperatur zwischen Vorderseite und Mitte des Absorbers \\
    ${t_{\mathrm{max}}}^{(1)}$       & Zeit bis zum maximalen Überschwingen eines PT2-Gliedes           \\
    ${t_{\mathrm{max}}}^{(2)}$       & Kritische Zeitspanne bis der Receiver beschädigt wird            \\
    $\TReturn{x}$                    & Temperatur der Rückführluft an der Stelle x                      \\
    $t_{\mathrm{settling}}$          & Ausregelzeit                                                     \\
    $T_s$                            & Abtastzeit                                                       \\
    $\boldsymbol{u}$                 & Stellgrößen                                                      \\
    $\UDotAbsorberBack$              & Änderung der inneren Energie der Absorberwabenrückseite          \\
    $\UDotAbsorberFront$             & Änderung der inneren Energie der Absorberwabenvorderseite        \\
    $u_{\mathrm{setpoint}}$          & Einstellwert der Gebläse/Ventil Kombination                      \\
    $\boldsymbol{w}$                 & Führungsgrößen                                                   \\
    $w_T$                            & Gewichtungsfaktor zur Bestimmung der mittleren Temperaturen      \\
    $\boldsymbol{W_{u}}$             & Gewichtungsmatrix für die Eingangsgrößen in der Kostenfunktion   \\
    $\boldsymbol{W_{w}}$             & Gewichtungsmatrix für die Regelabweichung in der Kostenfunktion  \\
    $\boldsymbol{W_{\xi}}$           & Gewcihtungsmatrix für die soft constraints in der Kostenfunktion \\
    $\boldsymbol{x}$                 & Modellzustände                                                   \\
    $x_{\mathrm{Cent}}$              & Schwerpunkt der Zielpunkte in x-Richtung  \\
    $\boldsymbol{y}$                 & Ausgangsgrößen                                                   \\
    $y_{\mathrm{Cent}}$              & Schwerpunkt der Zielpunkte in y-Richtung  \\
    $\boldsymbol{z}^{(1)}$           & Algebraische Größen                                              \\
    $\boldsymbol{z}^{(2)}$           & Störgrößen                                                       \\
\end{longtable}

\clearpage
\newpage \vspace*{-1cm}

\renewcommand{\arraystretch}{1.5}
\section*{Abkürzungen}
\begin{table}[ht!]
    \centering
\begin{tabular}{m{0.15\textwidth}m{0.75\textwidth}}
        \rowcolor{white}
Symbol & Bedeutung                                 \\
        \midrule
arr    & air return ratio \\
ASI     & All-sky imager                     \\
CSS     & Cloud Standby Szenario                    \\
DAE     & Algebraische Gleichung                    \\
DAPS     & Dynamic Aimpoint Processing System                     \\
DLR     & Deutsches Zentrum für Luft- und Raumfahrt                    \\
DNI     & Direct Normal Irradiation                     \\
EU     & Europäische Union                     \\
IPOPT     & Interior Point Optimizer                     \\
LQR     & Linear-Quadratischer Regler                    \\
MHE     & Moving Horizon Estimation                     \\
MIMO     & Multi Input Multi Output                     \\
MISO     & Multi Input Single Output                     \\
MPC     & Modellprädiktiver Regeler / Modellprädiktive Regelung                     \\
MPR     & Modellprädiktiver Regeler / Modellprädiktive Regelung                     \\
NLP     & Nichtlineares Programm                     \\
ODE     & Gewöhnliche Differentialgleichung                     \\
PID     & Proportional-Integral-Differential                     \\
RAM     & Random Access Memory                     \\
RMSE     & Root Mean Squared Error                     \\
SISO     & Single Input Single Output                     \\
STRAL     & Solar Tower Raytracing Laboratory                     \\
TH     & Technische Hochschule                     \\
    \end{tabular}
\end{table}
\clearpage
\newpage

