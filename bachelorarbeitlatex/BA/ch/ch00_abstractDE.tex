\chapter*{Kurzfassung / Abstract}
In dieser Arbeit wird die Modellbildung des Heliostatenfeldes und des Receivers des Solarturms in Jülich vorgestellt.
Außerdem wird eine modellprädiktive Regelung eingeführt, um die Luftaustrittstemperatur des Receivers unter Berücksichtigung zukünftiger Wolkenbedingungen zu regeln.
Die Stellgrößen des Reglers sind einerseits der Luftmassenstrom im Receiver sowie andererseits drei Streuungsfaktoren, die jedem Heliostaten ihren individuellen Zielpunkt auf dem Receiver zuweisen.
Die Strategie der Zielpunktverteilung basiert auf einer Veröffentlichung von García et al. \cite{Garcia2} und wurde differenzierbar approximiert.
Zur Bewertung der Regelung wurde ein Referenzszenario aus der Literatur eingeführt und verschiedene Wolkenszenarien definiert.
Die Güte der Regelung bemisst sich an der Abweichung der Luftaustrittstemperatur von der Referenztemperatur bei Nennlast.
Die Simulation der Regelstrecke zeigt, dass bei abweichungsfreier Vorhersage des Nowcastings eine Verringerung dieser Abweichung von bis zu $\SI{86.4}{\percent}$ erreicht wird.
Der Exergieeintrag in den Folgeprozess ist je nach Lichtdurchlässigkeit der Wolken um bis zu $\SI{19.2}{\percent}$ größer als im Referenzszenario.
Besonders bei geringer Verschattungsintensität ist die Regelung dabei sehr effektiv.
Die Betriebssicherheit des Kraftwerkes wird maßgeblich durch die Oberflächentemperatur des Receivers beeinflusst.
Eine Analyse fehlerhafter Wolkenprognosen zeigt, dass die Sicherheit für Vorhersagen der Wolkengeschwindigkeit von $\geq\SI{-11}{\percent}$ und für Vorhersagen der Lichtdurchlässigkeit von $\geq\SI{-5}{\percent}$ des Realwertes durch die Regelung gewährleistet ist.
Auf Basis dieser Arbeit kann die Regelung durch zusätzliche Testszenarien und alternative Software verbessert und an der Realanlage getestet werden.
