% !TeX encoding = utf8
% !TeX program = pdflatex
% !TeX spellcheck = de-DE
%\clearpage \thispagestyle{empty}
\chapter*{Abstract}
Dies soll eine Vorlage für den Einstieg in LaTeX darstellen.
Notwendig für die Nutzung dieser ist VSCode mit den entsprechenden Einstellungen in der settings.json, Strawberry Pearl sowie MikTex.

Generell ist das Hauptdokument die Main.tex. Die ist die einzige Datei, die letztlich ausgeführt wird. Alle anderen Dateien sollten in diese Datei eingebunden werden.

In der Documentclass in der Main.tex sowie in der Packages.tex entscheidet sich das Layout des Berichts. In der Misc.tex werden Informationen für die Titelseite hinzugefügt.
Es ist sehr empfehlenswert für jedes Kapitel eine neue *.tex Datei im Ordner \gans{ch} hinzuzufügen. Das Ergebnis der Kompilierung zeigt immer die Main.pdf.

Schreibt man einfach nur In VSCode in eine neue Zeile wird in der entstehenden PDF einfach in derselben Zeile weiter geschrieben.
Dennoch ist es sehr nützlich in neue Zeilen zu schreiben, sodass nachträglich zu korrigierende Sätze nicht in ewig langen Passagen untergehen. \\
Vor diesem Satz wurde nicht nur eine neue Zeile in VSCode eingefügt, sondern auch ein Doppel-Backslash \gans{\textbackslash\textbackslash} ans Ende der letzten Zeile gestellt. Dadurch entsteht eine neue Zeile ohne Abstand zum vorigen Abschnitt.

Lässt man eine Zeile in VSCode leer, entsteht ein wie oben gezeigter Abstand zwischen den verschiedenen Sätzen.\\

Lässt man eine Leerzeile und fügt sogar noch ein Doppel-Backslash \gans{\textbackslash\textbackslash} ans Ende des vorangestellten Abschnittes, so entsteht dieser Textabstand.

Gerade zu Beginn eines Kapitels oder Abschnittes hilft manchmal der \gans{\textbackslash noindent}-Befehl um das Einrücken zu Beginn einer Passage zu verhindern.

\cleardoublepage