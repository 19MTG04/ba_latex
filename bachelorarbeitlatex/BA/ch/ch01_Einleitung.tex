\chapter{Einleitung}\label{ch_ersteskapitel}
Die wichtigsten Befehle sind wohl die für die Einführung eines Kapitels \gans{\textbackslash chapter\{XX\}} oder \gans{\textbackslash section{}}. Generell beginnen auch alle anderen Befehle mit einem Backslash.

Das hier ist ein Kapitel. Der \textit{label} Befehl dient dazu, später auf das Kapitel zu referenzieren. Dies gilt auch für Sections, Subsections, Bilder, Tabellen, usw.\\

\section{Erstes Unterkapitel}
\label{sec_erstesunterkapitel}
Das hier ist ein Unterkapitel. Es befindet sich in Kapitel \ref{ch_ersteskapitel}. Mithilfe des \textit{ref}-Befehls wird auf das Label des Kapitels Bezug genommen.
Die Nummerierung geschieht dann Latex-intern von alleine.


\subsection{Unterkapitel vom Unterkapitel}
\label{subsec_unterkapitelvomunterkapitel}
Das hier ist ein Unterkapitel vom Unterkapitel und befindet sich in Kapitel \ref{sec_erstesunterkapitel}.

\subsection{Grafiken}
\label{Grafiken}
Eine Grafik wird wie folgt eingebunden:\\
\begin{figure}[h!]
    \centering
    \setlength{\fboxsep}{1pt}
    \setlength{\fboxrule}{1pt}
    \fbox{\includegraphics[width=0.45\textwidth]{fig/ch02_beispielBild.png}}
    \caption[EintragAbbildungsverzeichnis]{Bildunterschrift}
    \label{Label}
\end{figure}

Oben steht nun Abbildung \ref{Label}

Anmerkung: Lässt man die eckigen Klammern bei der Caption weg, so wird im Abbildungsverzeichnis automatisch der Inhalt der geschweiften Klammern übernommen.
Zweite Anmerkung: Der Befehl \textit{fbox} setzt einen Rahmen um das Bild. Kann natürlich auch entfernt werden. Die Abbildung wird automatisch im Abbildungsverzeichnis eingetragen!

\subsection{Formeln}
\label{Formeln}

Eine Formel geht wie folgt:\\
\begin{equation} \label{eq_Label}
    a^2+b^2
\end{equation}
\centerline{\textbf{Formel \ref{eq_Label}:} Beschriftung}
\myequations{EintragFormelverzeichnis}

Auch die Formel \ref{eq_Label} wird auf diese Weise automatisch in das Formelverzeichnis übernommen.

\subsection{Tabellen}
\label{Tabellen}
Eine Tabelle wird zum Beispiel über ein Excel-Tool eingefügt. Der Link dazu ist: \url{https://ctan.org/pkg/excel2latex?lang=de}.
Dies kann dann in Excel aktiviert werden und Tabellen extrahiert werden.

%%%%%%%%%%%%%%%%%%%%%%%%%%%%%%%%%%%%%%%%%%%%%%%%%%%%%%%%%%%%%%%%%%%%%%%%%%%%%%%%%%%%
%Die nachfolgende Tabelle wurde automatisch über das Latex2Excel-Tool erstellt.
\begin{table}[htbp]
    \centering
    \caption{Erste Tabelle}
    \begin{tabular}{lrll}
        \toprule
        \multicolumn{1}{c}{\textbf{Name}}          & \multicolumn{1}{c}{\textbf{Alter}} & \multicolumn{1}{c}{\textbf{Stadt}} & \multicolumn{1}{c}{\textbf{Land}} \\
        \midrule
        \rowcolor[rgb]{ .851,  .851,  .851} Tobias & 23                                 & Leverkusen                         & Deutschland                       \\
        Julia                                      & 19                                 & Wuppertal                          & Deutschland                       \\
        \rowcolor[rgb]{ .851,  .851,  .851} David  & 19                                 & Leverkusen                         & Deutschland                       \\
        \bottomrule
    \end{tabular}%
    \label{tab_erste Tabelle}%
\end{table}%
%%%%%%%%%%%%%%%%%%%%%%%%%%%%%%%%%%%%%%%%%%%%%%%%%%%%%%%%%%%%%%%%%%%%%%%%%%%%%%%%%%%%
Die Tabelle \ref{tab_erste Tabelle} wird automatisch im Tabellenverzeichnis eingetragen!

\subsection{Fußnoten}
\label{subsec_Fußnoten}
Wenn man einen Text schreibt, dann kann man z.B. hier eine Fußnote\footnote{Und dann steht hier eine Zusatzinformation} einfügen. Andere nette Spielerin sind das einfügen von Links wie der hier: \url{https://www.wwf-jugend.de/blogs/5830/5830/unbekannter-und-bedrohter-fisch-der-blobfisch}.


\subsection{Zitieren}
\label{subsec_Zitieren}

So macht man ein Zitat \cite{Follinger.2016}. Oder so \cite{ifmelectronic.2004}\\
Dafür wird in diesem Falle Citavi benutzt. In einem Citavi Projekt sollte für LaTeX-Vorlagen dieser Art immer folgendermaßen eine Quelle eingebunden werden:
\begin{itemize}
    \item Auf das $+$ klicken, um einen Titel hinzuzufügen.
    \item Unabhängig von der realen Art der Quelle sollte IMMER \glqq Buch (Monographie)\grqq{} oder wenn nötig \glqq Buch (Sammelwerk)\grqq{} gewählt werden.
    \item Autor, Titel sowie Online-Adresse und ISBN wenn vorhanden angeben. Zuletzt geprüft am: Sollte z.B. mit: \glqq 23.07.2021 um 12:44 Uhr\grqq{} ausgefüllt werden.
    \item In Citavi kann unter Datei, Exportieren, das Format \glqq BibTeX\grqq{} gewählt werden (LaTeX Notation verwenden) und anschließend als .bib Datei gespeichert werden.
    \item Diese Datei muss anschließend im LaTeX Programm hochgeladen werden. Im Main Dokument darauf achten, dass die Bezeichnung der bib Datei richtig referenziert wird.
\end{itemize}
Mit ein bisschen Recherche kann alles so angepasst werden wie man es braucht. Einfach mal Googeln oder jemanden fragen.\\
Das wars. Viel Spaß.
