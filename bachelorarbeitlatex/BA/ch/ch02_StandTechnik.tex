\chapter{Stand der Technik} \label{ch_StandTechnik}
Hier wird der Stand der Technik eingeleitet.

\section{Solartechnik} \label{sec_Solartechnik}
Alle Themen-relevanten solartechnischen Unterkapitel.

\subsection{Solartürme} \label{subsec_Solartürme}
Was gibt es bereits, was sind aktuelle Herausforderungen?
Wie groß ist der Anteil am EE Mix?
Auf verschiedene Standorte eingehen etc.

\subsection{Receiver} \label{subsec_Receiver}
Eingehen was für verschiedene Receiver es gibt und dass es sich bei diesem um einen keramischen Receiver handelt.
Was sind die Vor- und Nachteile dieses Receivers?
Was sind die Temperaturbedingten Grenzen unseres Receivers?

\subsection{Weitere Kraftwerkskomponenten} \label{subsec_WeitereKomponenten}
Weitere relevante Komponenten wie den Kraftwerksprozess oder den Wärmespeicher sowie besonders auch die Heliostate.
Heliostate als eigene Überschrift?

\section{Modellprädiktive Regelung} \label{sec_ModellprädiktiveRegelung}
Wichtigste Begriffe und Funktionalität von MPC einbringen.
Auf jeden Fall auch entsprechende Abbildungen rein.
Hier soll schon das Alleinstellungsmerkmal der Prädiktion mit rein.

\subsection{Solver} \label{subsec_Solver}
Hier ggf. den verwendeten Solver bzw. das Buch dazu eingehen? \cite{Follinger.2016}

\subsection{Robuste MPC} \label{subsec_RobusteMPC}
Was unterscheidet robuster von \gans{normaler} MPC? Was sind Vor- und Nachteile?
Warum wird es hier nicht benutzt?

\section{Zielpunktregelung} \label{sec_Zielpunktregelung}
Was macht de Zielpunktregelung die wir nutzen aus?
Was unterscheidet Sie von anderen Zielpunktregelungen?
Warum haben wir uns dafür entschieden (Stellgrößenreduktion bei Garcia)?
Warum haben wir uns für die Überlagerung der Flussdichtekarten entschieden?

\section{Nowcasting-Systeme zur Wettervorhersage} \label{sec_Nowcasting}
Welches Nowcasting-System verwendet das DLR und warum?
Wie ist dort generell der Stand der Technik?
Was ist die Auflösung bzw. andere Besonderheiten dieses Nowcasitng-Systems?

\section{Neuronale Netze} \label{sec_NN}
Was macht ein Neuronales Netzwerk aus und wie funktioniert es?
Warum brauchen wir es, was machen wir damit (Extrahierte Funktion differenzierbar)?
Wie sieht ein Trainingsprozess aus und warum hilft uns das?
Quellen: http://www.informatikseite.de/neuro/node16.php (https://the-decoder.de/kuenstliche-neuronale-netze-erklaerung/).
Diesen Abschnitt, wenn überhaupt, so kurz wie möglich halten.

