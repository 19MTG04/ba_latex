\chapter{Modellbildung} \label{ch_Modellbildung}
Bisschen drauf eingehen was jetzt als nächstes kommt.
Erst das alte und dann das kombinierte System darstellen.



\section{Erweiterung der Modellbildung} \label{sec_ErweiterungModellbildung}
Hier komtm dann alles bezüglich der Erweuterung des thermischen Modells hin zb um den Fan und dann das Coupling mit dem optischen Modell.
Dafür habe ich ja schon all die Bilder vorbereitet.


Hier auch hinschreiben, dass linear approximiert wurde

Warum haben wir uns für die Überlagerung der Flussdichtekarten entschieden?
Wie groß ist der prozentuale Fehler durch Vorkalkulation in der Mitte und Verschiebung zum Zielpunkt im Vergleich zur direkten Simulation an der richtigen Stelle?

Also: Alles was ich an dem Modell (Standard-Modell mit nur einem Absorber-Cup) verändert habe.

\subsection{Implementierung der Lüftungs-Dynamik} \label{subsec_ImplementierungFan}
Mit dem Parameter Fitting und so.
Grafik wo die Messkurve und die Simulationskurve übereinander liegen.
Erklärung der PT2-Werte und deren Herleitung?

\subsection{Implementierung des optischen Modells} \label{subsec_Modells}
Alles bezüglich der Heliostaten und deren Gruppierung (Warum 20x20, wegen NowCasting) sowie der Fluxmap 12x10 Verteilung.
Erläutern, dass direktes Mapping auf die Flussdichte nicht funktionieren kann?
Daher Berechnung der Flussdichteverteilung über gruppierte Zielpunktregelung, Wolken können am einfachsten implementiert werden.

Vorberechnete Strahlungskarten verwendet.
Welche Fehler wurden dabei berücksichtigt? David fragen!!