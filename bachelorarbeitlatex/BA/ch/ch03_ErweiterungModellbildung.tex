\chapter{Modellbildung} \label{ch_Modellbildung}
Bisschen drauf eingehen was jetzt als nächstes kommt.
Erst das alte und dann das kombinierte System darstellen.

% Vorher schon beschrieben und von David hierher gewollt!
Für die Analyse des Systems wird nachfolgend mit vorberechneten Strahlungskarten aus dem in \cite[S.53ff]{DissBelhomme} vorgestellten Programm \textit{STRAL} verwendet.
Die Software berücksichtigt das reale Heliostatenfeld am Standort Jülich und bietet auch die Möglichkeit optische Verluste zu einem gewissen Grad mit einzubeziehen.

% Schon geschrieben: Welchen Regelalgorithmus wähle ich warum?!
Die Algorithmen, die nicht auf Basis einer Gruppierung von Heliostaten funktionieren, haben einen hohen Rechen- und Zeitaufwand in der Optimierung.
Weiterhin ist der DAPS-Algorithmus aufgrund anderer Nachteile für diese Arbeit ungeeignet.
Dazu zählt, dass er nicht für die Inbezugnahme von Wolken ausgelegt ist und keine Leistungsmaximierung geschieht, sondern lediglich die Einhaltung der Grenztemperaturen gewährleistet wird.
Darüber hinaus besteht die Notwendigkeit einer sehr präzisen Modellbildung, um den exakten Einfluss einzelner Heliostaten nutzen zu können.
Weiterhin wird immer der Heliostat mit dem größten Einfluss manipuliert, welcher nicht zwangsläufig der ideale Heliostat in Bezug auf die Optimierung ist. \cite[S.35]{DissOberkirsch}
Die Einteilung des Systems in SISO-Subsysteme ist für stark gekoppelte Systeme mit großen Abhängigkeiten nicht sinnvoll. \cite[S.33]{DissZanger}

% Schon geschrieben: Änderungen an dem Algorithmus
Für die vorliegende Arbeit entfällt die Unterteilung in 18 Gruppen aufgrund der Receiver- und Heliostatenfeldstruktur, sodass das gesamte Feld in lediglich drei Teile eingeteilt wird.
Weiterhin unterscheiden sich in dieser Ausarbeitung auch die erste und zweite Gruppe in der Distanz bezogen auf den Receiver, wie in Kapitel \ref{sec_ErweiterungModellbildung} erläutert wird.

Dann auch rein mit der linearen Approximierung.


\section{Erweiterung der Modellbildung} \label{sec_ErweiterungModellbildung}
Hier komtm dann alles bezüglich der Erweuterung des thermischen Modells hin zb um den Fan und dann das Coupling mit dem optischen Modell.
Dafür habe ich ja schon all die Bilder vorbereitet.


Hier auch hinschreiben, dass linear approximiert wurde

Warum haben wir uns für die Überlagerung der Flussdichtekarten entschieden?
Wie groß ist der prozentuale Fehler durch Vorkalkulation in der Mitte und Verschiebung zum Zielpunkt im Vergleich zur direkten Simulation an der richtigen Stelle?

Also: Alles was ich an dem Modell (Standard-Modell mit nur einem Absorber-Cup) verändert habe.

\subsection{Implementierung der Lüftungs-Dynamik} \label{subsec_ImplementierungFan}
Mit dem Parameter Fitting und so.
Grafik wo die Messkurve und die Simulationskurve übereinander liegen.
Erklärung der PT2-Werte und deren Herleitung?

\subsection{Implementierung des optischen Modells} \label{subsec_Modells}
Alles bezüglich der Heliostaten und deren Gruppierung (Warum 20x20, wegen NowCasting) sowie der Fluxmap 12x10 Verteilung.
Erläutern, dass direktes Mapping auf die Flussdichte nicht funktionieren kann?
Daher Berechnung der Flussdichteverteilung über gruppierte Zielpunktregelung, Wolken können am einfachsten implementiert werden.

Vorberechnete Strahlungskarten verwendet.
Welche Fehler wurden dabei berücksichtigt? David fragen!!