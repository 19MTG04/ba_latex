\chapter{Reglerentwurf} \label{ch_Reglerentwurf}
Hier kommt alles über das Buch von Davids Einleitung rein, zumindest mal die Unterpunkte von 2-4 ganz explizit.

\section{Anpassung des bestehenden Modells} \label{sec_AnpassungModell}
Alles was ich an dem Modell (Standard-Modell mit nur einem Absorber-Cup) verändert habe.

\subsection{Implementierung der Lüftungs-Dynamik} \label{subsec_ImplementierungFan}
Mit dem Parameter Fitting und so.
Grafik wo die Messkurve und die Simulationskurve übereinander liegen.
Erklärung der PT2-Werte und deren Herleitung?

\subsection{Implementierung der Heliostaten als Stellgrößen} \label{subsec_ImplementierungHeliostate}
Alles bezüglich der Heliostaten und deren Gruppierung (Warum 20x20, wegen NowCasting) sowie der Fluxmap 12x10 Verteilung.
Erläutern, dass direktes Mapping auf die Flussdichte nicht funktionieren kann?
Daher Berechnung der Flussdichteverteilung über gruppierte Zielpunktregelung, Wolken können am einfachsten implementiert werden.

\section{Regelgrößen} \label{sec_Regelgrößen}
Das sind die dann wohl die States aber siehe Davids Literatur aus seiner Einleitung.
Warum sind diese Größen die Regelgrößen?

\section{Stell- und Messgrößen} \label{sec_StellMessgrößen}
Auch hier siehe Davids Buch.
Warum entscheiden wir uns für die entsprechenden Größen als Mess- und Stellgrößen?

\section{Auslegung des Modellprädiktiven Reglers} \label{sec_AuslegungMPC}
Hier wird dann letztlich genau alles über den Regler und die Objective sowie die Constraints und alles was an Einstellungen getroffen wurde und warum dargestellt inkl. Schrittweite und Prädiktionshorizont. Für Schrittweite auch Negativbeispiel anfügen mit tstep = 15! Für Prädiktionshorizont Systemdynamik und eigentliche Auslegung darstellen und dann Berechnungszeiten miteinander vergleichen und sagen, warum 6 gewählt wurde.

