\chapter{Reglerentwurf} \label{ch_Reglerentwurf}
Hier kommt alles über das Buch von Davids Einleitung rein, zumindest mal die Unterpunkte von 2-4 ganz explizit.

% Vorher bereits geschrieben und auf Methodeneteil verwiesen
Im Anwendungsfall dieser Arbeit ist dabei von besonderer Bedeutung, dass auch zukünftige Modellparameter in die Regelung prädiktiv einbezogen werden können.
Aus diesem Grund sind klassische Regler wie der PID-Regler, die lediglich auf Basis der Abweichung von aktuellen Soll- und Istwerten Stellgrößen für das System vorgeben \cite[S.408]{Lunze}, nicht ausreichend und es wird ein modellprädiktiver Regler (kurz \textit{MPC}) gewählt.

Hier soll die Auswahl der Regelungsmethode und der Kollokationsmethode mit Radau kommen.

\section{Regelgrößen} \label{sec_Regelgrößen}
Das sind die dann wohl die States aber siehe Davids Literatur aus seiner Einleitung.
Warum sind diese Größen die Regelgrößen?

\section{Stell- und Messgrößen} \label{sec_StellMessgrößen}
Auch hier siehe Davids Buch.
Warum entscheiden wir uns für die entsprechenden Größen als Mess- und Stellgrößen?

\section{Auslegung des Modellprädiktiven Reglers} \label{sec_AuslegungMPC}
Hier wird dann letztlich genau alles über den Regler und die Objective sowie die Constraints und alles was an Einstellungen getroffen wurde und warum dargestellt inkl. Schrittweite und Prädiktionshorizont. Für Schrittweite auch Negativbeispiel anfügen mit tstep = 15! Für Prädiktionshorizont Systemdynamik und eigentliche Auslegung darstellen und dann Berechnungszeiten miteinander vergleichen und sagen, warum 6 gewählt wurde.

Sagen welche collocation method benutzt wurde und warum.
