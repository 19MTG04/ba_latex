\chapter{Analyse der Modellprädiktiven Regelung} \label{ch_AnalyseRegelung}
Hier sollen alle Ergebnisse hin.
Es soll aus der Zielsetzung klar werden, dass das das Kapitel ist, auf das es ankommt und dass hier alle meine Ergebnisse stehen.
Die Ergebnisse müssen dann am Ende aber natürlich auch das aussagen, was ich aussagen möchte und vernünftig analysiert und ausgewertet werden.

\section{Repräsentative Wolkenfälle} \label{sec_Wolkenfälle}
Was ist das und warum ist das wichtig?
Für welche habe ich mich entschieden, um was genau zu analysieren?

Es muss immer klar sein, ob der MPC gerade von den Ergebnissen weiß oder nicht weiß.
Ggf. nochmal aufzeigen, warum die Einstellungen so sind wie sie sind und wann eine Regelung als gescheitert gilt?
Es muss am Ende rausgestellt werden, was die Grenzen der MPC sind und wie viel besser es war, im Vergleich zu einer unwissenden Regelung.

\section{Einfluss der Wolkengröße} \label{sec_EinflussGröße}
Hier soll rein welchen Einfluss die Wolkengröße hat.
Wie präzise kann der MPC dennoch die Temperatur halten?

\section{Einfluss der Wolkengeschwindigkeit} \label{sec_EinflussGeschwindigkeit}
Spätestens hier soll erläutert werden, welche Wolkengeschwindigkeiten in der Realität auftreten und wie sie mit in unser System reinspielen.
Was macht die höhere Wolkengeschwindigkeit? Warum ist die hohe Geschwindigkeit gefährlich?
Weil der MPC gegebenenfalls gar nicht weiß, dass gerade das ganze Feld wieder sonnig ist aufgrund seiner Sample Time!

\section{Einfluss der Lichtdurchlässigkeit} \label{sec_EinflussLichtdurchlässigkeit}
Hier soll rauskommen, dass die Lichtdurchlässigkeit, wenn der MPC davon weiß natürlich dazu beiträgt wie gut die Temperatur gehalten werden kann, aber auch dass in Kombination mit schnellen Wolken dieser Wert sehr wichtig ist, damit die Grenzen eingehalten werden.
Wie falsch darf die Vorhersage prozentual je nach Wolkengeschwindigkeit sein, damit es keine Probleme gibt?

\section{Einfluss der Wolkenrichtung} \label{sec_EinflussRichtung}
Hier muss rauskommen, warum sich das System anders verhält, wenn die Wolken in andere Richtungen ziehen (Weil die vorderen Heliostate natürlich deutlich mehr Einfluss auf das Ergebnis haben).

