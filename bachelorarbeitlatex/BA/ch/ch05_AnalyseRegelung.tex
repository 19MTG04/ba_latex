\chapter{Analyse der Modellprädiktiven Regelung} \label{ch_AnalyseRegelung}
Hier sollen alle Ergebnisse hin.
Es soll aus der Zielsetzung klar werden, dass das das Kapitel ist, auf das es ankommt und dass hier alle meine Ergebnisse stehen.
Die Ergebnisse müssen dann am Ende aber natürlich auch das aussagen, was ich aussagen möchte und vernünftig analysiert und ausgewertet werden.

Erstmal etwas Feldanalyse wie staedy state Antwort und die Vergleiche zwischen rechts, links, oben und unten.
Refernzszenario ganz wichtig.

Spätestens hier sollte auch aus Davids Intro Presi die Folie 5 rein.

Also hier soll das Systemverhalten beschrieben werden, dann soll gezeigt werden wie es ist, wenn der Regler von allem weiß, dann wenn er von nichts weiß, dann erst geht es um die fehlerhaften Informationen.
Dabei wird dann speed, shading und Richtung untersucht.

% VON VORHER UND IN KORREKTUR HIERHER VERSCHOBEN:
Die vorliegenden Einschränkungen bezüglich thermischen Maximalbelastungen des Receivers werden dem Betriebshandbuch entnommen.
Dort wird die maximale lokale Temperatur der Absorberfläche, welche direkt von der Flussdichte $\phi$ abhängig ist, mit $\SI{1275.15}{\kelvin}$ angegeben \cite{HandbuchJülich}. Dieser Wert dient in den Simulationen als Indikator für eine erfolgreiche Regelung, welche die Einhaltung dieses Limits als Grundvoraussetzung beinhaltet.

Vergleich vorlegen in der Regelung mit Szenario, wenn keine Regelung vorliegt! (cloud standby)
Außerdem auch die Enthalpien darstellen, wie sie als Ergebnis in dem process result data Skript herauskommen.
Dabei ist sowohl der Enthalpiestrom als auch die gesamte ins System aufgenommene Energie als integrale Größe interessant.
Diese kommen aus dem process_result_data Skript und der Funktion calculate_enthalpy_flow_from_simulation_data.

% Von vorher und hierher verschoben
Die gesamte Modellbildung, wie sie in Kapitel \ref{ch_Modellbildung} beschrieben ist, wird in do-mpc realisiert und ausgewertet.
Für die Simulation der Ergebnisse wird in dieser Arbeit ein 4-Kern Intel Core i7-1185G7 Prozessor mit einem Basistakt von 3,0 GHz verwendet.
Als Arbeitsspeicher stehen 16 GB RAM mit 4267 MHz zur Verfügung.
Das Betriebssystem ist Windows 10 Enterprise (Version 21H2).

\section{Repräsentative Wolkenfälle} \label{sec_Wolkenfälle}
Was ist das und warum ist das wichtig?
Für welche habe ich mich entschieden, um was genau zu analysieren?

Es muss immer klar sein, ob der MPC gerade von den Ergebnissen weiß oder nicht weiß.
Ggf. nochmal aufzeigen, warum die Einstellungen so sind wie sie sind und wann eine Regelung als gescheitert gilt?
Es muss am Ende rausgestellt werden, was die Grenzen der MPC sind und wie viel besser es war, im Vergleich zu einer unwissenden Regelung.

Daten von Nouri Bijan.



\section{Einfluss der Wolkengeschwindigkeit} \label{sec_EinflussGeschwindigkeit}
Spätestens hier soll erläutert werden, welche Wolkengeschwindigkeiten in der Realität auftreten und wie sie mit in unser System reinspielen.
Was macht die höhere Wolkengeschwindigkeit? Warum ist die hohe Geschwindigkeit gefährlich?
Weil der MPC gegebenenfalls gar nicht weiß, dass gerade das ganze Feld wieder sonnig ist aufgrund seiner Sample Time!

\section{Einfluss der Lichtdurchlässigkeit} \label{sec_EinflussLichtdurchlässigkeit}
Hier soll rauskommen, dass die Lichtdurchlässigkeit, wenn der MPC davon weiß natürlich dazu beiträgt wie gut die Temperatur gehalten werden kann, aber auch dass in Kombination mit schnellen Wolken dieser Wert sehr wichtig ist, damit die Grenzen eingehalten werden.
Wie falsch darf die Vorhersage prozentual je nach Wolkengeschwindigkeit sein, damit es keine Probleme gibt?

\section{Einfluss der Verschattungsdauer} \label{sec_Verschattungsdauer}
Nachträglich eingefügt, ggf sinnvoll?

\section{Einfluss der Wolkengröße} \label{sec_EinflussGröße}
Hier soll rein welchen Einfluss die Wolkengröße hat.
Wie präzise kann der MPC dennoch die Temperatur halten?

\section{Einfluss der Wolkenrichtung} \label{sec_EinflussRichtung}
Hier muss rauskommen, warum sich das System anders verhält, wenn die Wolken in andere Richtungen ziehen (Weil die vorderen Heliostate natürlich deutlich mehr Einfluss auf das Ergebnis haben).

Hier vllt eine Wolke einführen die ganz genau so groß wie das Feld ist.
Wenn die Wolke schräg verläuft, wird das Feld weniger stark verschattet.
Ergebnis soll eine Art umgekehrte Gaus Funktion sein wenn der RMSE über der vorhergesagten Richtung (-45 bis 45 Grad, die eigentlich 00 gerade nach oben ist) aufgetragen wird.


Generell muss erläutert werden, auf welcher Maschine die Analysen laufen.
Begründen, warum für die Szenarien die Bewegung von unten nach oben genommen wird: Dadurch wird bei plötzlicher Belichtung des Feldes der Receiver der maximalen Belastung ausgesetzt, da die Heliostaten unten die meiste Leistung übertragen.

% Wichtig:
Analyse muss sich in den Details mit dem decken, was David in dem Paper mit meinem Namen darauf sagt.
Besonders auch, dass der Simulator mit anderen Flussdichtekarten rechnet als der Controller.
