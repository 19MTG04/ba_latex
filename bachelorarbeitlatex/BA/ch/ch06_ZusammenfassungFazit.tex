\chapter{Zusammenfassung und Fazit} \label{ch_Fazit}
In dieser Arbeit wurde ein thermisches Teilmodell zur Beschreibung der Luftaustrittstemperatur des Receivers am Solarturm in Jülich vorgestellt und um die Gebläsedynamik erweitert.
Zur Beschreibung der Zielpunktverteilung und des thermischen Einflusses auf den Receiver wurden weiterhin zwei optische Modelle entwickelt.
Eines dieser Modelle dient dem Optimierungsprozess der Regelung und arbeitet mit einer simulierten Wolkenvorhersage, die in der Realität durch ein Nowcasting-System ersetzt werden kann.
Das andere optische Modell wurde zur Simulation des Solarturms entwickelt.
Die optischen Modelle wurden mit dem thermischen Modell zu zwei verschiedenen Gesamtmodellen kombiniert.

Weiterhin wurde ein modellprädiktiver Regler ausgelegt und mit den Modellen zu einem Regelkreis verknüpft.
Für verschiedene Wolkenszenarien wurde die Regelgüte bezüglich der Abweichung der Luftaustrittstemperatur von dem entsprechenden Sollwert im Vergleich zu einem Referenzszenario bestimmt.
Zusätzlich wurde auch der Einfluss der Regelung auf den Exergieeintrag ins System analysiert.
Durch die Regelung mit abweichungsfreier Wolkenprädiktion kann die Temperaturabweichung um $\SIrange{26.0}{86.4}{\percent}$ reduziert werden, wobei besonders für Wolkenszenarien mit moderater Abschattung gute Regelungsergebnisse erzielt werden.
Der Exergieeintrag, der nicht direkt als Optimierungsgröße in die Kostenfunktion des Reglers einbezogen wird, schwankt gegenüber dem Referenzszenario zwischen $\SI{-0.8}{\percent}$ für vollständige Abschattung und $\SI{+19.2}{\percent}$ bei hoher Lichtdurchlässigkeit der Wolken.

Zur Bewertung der Robustheit der Regelung gegenüber fehlerhafter Wolkenprädiktionen wurden separate Analysen für abweichend vorhergesagte Wolkengeschwindigkeiten und Verschattungsintensitäten durchgeführt.
Diese ergaben, dass die Vorhersage bezüglich der Wolkengeschwindigkeit eine Abweichung von $\SI{-11}{\percent}$ des Realwertes nicht unterschreiten darf, um die Betriebssicherheit des Kraftwerkes nicht zu gefährden.
Die Prädiktion der Lichtdurchlässigkeit darf den Realwert um maximal $\SI{-5}{\percent}$ unterschreiten.
Schnellere Vorhersagen der Wolkengeschwindigkeit oder der Lichtdurchlässigkeit gefährden die Sicherheit des Receivers nicht.

Insgesamt konnte der vorgestellte Regler die Zielsetzung der Arbeit erfüllen.
So wird durch das prädiktive Regelverhalten die Abweichung zwischen der Austrittstemperatur und dem Sollwert gegenüber dem Referenzszenario reduziert.
Dementsprechend wird der Wirkungsgrad des Solarturmkraftwerkes unter Wolkendurchzug verbessert sowie die Lebensdauer der Komponenten erhöht.
Zusätzlich wird unter der aufgezeigten Toleranz der Genauigkeit des Nowcastings die Temperaturbegrenzung des Receivers nicht überschritten.

Wie in Kapitel \ref{sec_Hardsoftanalyse} erwähnt ist es mit der genutzen Hard- und Software nicht problemlos möglich, die Berechnung zu jedem Abtastzeitpunkt der Optimierung bis zur Konvergenz durchzuführen, da sonst Berechnungsdauern oberhalb der Abtastzeit erreicht werden.
Eine Lösung dieses Problems könnte die Nutzung alternativer Solver sein.
Auf der Grundlage dieser Arbeit können auch weitere Arbeiten zur Verbesserung der Regelung eines Solarturms durchgeführt werden.
Einige dieser Ansätze zur weiteren Forschung werden nachträglich vorgestellt.

\subsubsection*{Weitere Testszenarien}
Der Regler wurde nur für einige Testfälle zur Wolkengeschwindigkeit und Verschattungsdauer getestet und kann daher in anderen Szenarien mit längerer Verschattungszeit oder anderer Wolkenrichtung womöglich schlechtere Ergebnisse liefern.
Auch die Robustheit der Regelung bezüglich dieser Größen bei fehlerhafter Prädiktion des Nowcastings ist zu prüfen.
Weiterhin wurde kein Test vorgenommen, der für mehrere gleichzeitig fehlerhaft vorhergesagte Parameter die Robustheit prüft.

\subsubsection*{Robuste MPC}
Für Regelstrecken mit vorhergesagten Parametern ist die Implementierung einer robusten modellprädiktiven Regelung womöglich sinnvoll.
Die Grundidee besteht darin, abhängig von als \gans{unsicher} deklarierten Variablen, verschiedene Modelle oder Modellparameter für die Berechnung in jedem Abtastzeitpunkt zu nutzen.
Auf diese Weise steigt die Robustheit gegenüber Fehlern der Prädiktion, da die Optimierung bereits für verschiedene Szenarien der unsichereren Parameter durchgeführt wurde.
Dies ist jedoch mit einem erhöhten Rechenaufwand verbunden \cite[S.6ff]{Schwenzer}.

\subsubsection*{Implementierung der Regelung an der Realanlage}
Die Validierung der Forschungsergebnisse dieser Arbeit soll am Solarturm in Jülich geschehen.
Notwendige Teilschritte diesbezüglich wurden bereits in Kapitel \ref{sec_RealsystemRegelung} aufgeführt.
Bei erfolgreicher Testung der Regelung kann diese für die allgemeinere Nutzung mit verschiedenen Receiver- und Heliostatentypen generalisiert werden.

Zusammenfassend lässt die vorgestellte Regelung als vielversprechend bewerten.
Allerdings bestehen noch viele Möglichkeiten für zukünftige Arbeiten zur Verbesserung und Erweiterung der Funktionalität.
