\chapter{Method}

% Describe the building blocks of the receiver in Jülich.
This work models the open-volumetric receiver at the solar tower in Jülich.
The receiver front consists of individual absorber cups.
These absorber cups are further divided into four sectors.
% NOTE: Why is the temperature homogenized (reasoning by Gall)? What is the actual reason for the subreceivers?
Each sector has its own header element, which mixes the air flows through the cups to homogenize the temperature.
Behind the header a valve is placed, to regulate the mass flow through each sector.
% NOTE: Is that true?
In practice these valves are all open.
After the mixture in the four primary headers, the resulting air flows are further combined in a secondary header.
The out flowing air of the secondary header is then directed to either the heat storage or the power block.
The cooled but still warm (usually \SIrange{80}{120}{\degreeCelsius}) returning air is redirected to the absorber cups.
The air is blown through gaps between the absorber cups in front of the receiver and is recirculated there with ambient air.
In addition to cooling the structure, this also increases the efficiency of the absorber.

% Describe the assumptions made to simplify the model.

To develop a fast dynamical model, the model highly simplified using the following assumptions:

\begin{itemize}
% NOTE: Can we give an example for the validity of this assumption?
    \item Only temperatures are used as state variables, pressure is not used due to the much faster dynamics of the pressure.
    \item A constant pressure is assumed (isobaric), so that the enthalpy of the air is determined by the temperature alone.
    \item Heat transfer occurs exclusively in the radial direction. Thus, there no heat exchanges between the individual absorber cups.
    \item The temperatures of the return air along the absorber and of the heated air behind the absorber comb are homogeneous. This assumption simplifies the calculation of the heat transfer between the two fluids and is nevertheless sufficiently accurate, since the local temperature change is relatively small.
    \item The air temperature is homogeneous within balance spaces and changes discretely at balance space boundaries.
    \item The absorber comb is the only component that stores thermal energy.
\end{itemize}

To model the receiver, the receiver is divided into the absorber cups and the headers.

\section{Absorber Cups}

As stated before, the absorber cups do not exchange heat between each other.
Thus, it is sufficient to consider one absorber cup first and construct the whole receiver model out of many individual cups.

% NOTE: Is absorber tub a good translation for Absorberkelch?
% Using absorber cup for this word might be confusing.
A single absorber consists of three components: absorber comb, absorber tub and a piece of tubing.
The absorber comb consists of a porous ceramic that absorbs the solar radiation, converts it into thermal energy and stores it as internal energy.
The air flowing through the porous structure acts as a heat transfer medium.
It absorbs the energy, transferred by the ceramic and transports it via the air circuit.
The absorber tub has roughly the shape of a truncated pyramid with a square base and a round outlet.
It is solid and made of the same material as the comb.
From the cylindrical tube, the air flows into the headers.
The first part of the tube, in addition to insulation, consists of the end piece of the tub, which must be taken into account when calculating the heat transfer.

\textbf{Energy balances}
\cref{???} illustrates the energy balances of a single absorber.
The absorber cup is discretized into a heating zone and a transport zone.
The heating zone represents the absorber comb, which absorbs the solar irradiation.
\TODO{Sketch both pipe sections in the figure.}
The transport zone consists of the absorber tub and the two pipe sections.
It would also be possible to treat each part as an individual balance space.
However, since the temperature does not change along these components, one balance space combines these three parts.

\textbf{Heating zone}
The energy balance around the heating zone (see \cref{?}) yields the following equations

% Describe the basic energy balances for the absorber cup.
\begin{equation}
\UDotAbsorber =\QDotSol - \QDotLossAbsorberCupConvective -  \QDotLossAbsorberCupRadiative -\QDotAbsorberComb
\label{eq:UDotAbsorber}
\end{equation}
Here, \(\UDotAbsorber\) describes the change in internal energy of the comb, \(\QDotAbsorberComb\) the heat transfer between the absorber comb and the incoming air and \(\QDotSol\) the solar irradiation absorbed by the comb.
The loss terms \(\QDotLossAbsorberCupConvective\) and \(\QDotLossAbsorberCupRadiative\) represent the thermal loss due to wind and radiation respectively.
The internal energy rate can be expressed as
\begin{equation}
    \UDotAbsorber = m_{\mathrm{abs}}  c_{\mathrm{abs}} \frac{d \TAbsorber}{d t}
\end{equation}
with \(m_{\mathrm{abs}}\) and \(c_{\mathrm{abs}}\) describing the absorber mass and heat capacity respectively.

\(\QDotSol\) can be stated as
\begin{equation}
\QDotSol = \alpha_{\mathrm{sol}} \PowerSolar =  \epsilon \SurfaceFront \FluxSolar
\end{equation}
with \(\alpha_{\mathrm{sol}}\) being the absorption coefficient and \(\SurfaceFront\) the absorbing surface of the absorber comb.
\(\FluxSolar\) is the flux density (solar irradiation per area) on the absorber comb, which can be measured for example by the system developed by \cref{Mathias Offergeld}.

The loss due to convection \(\QDotLossAbsorberCupConvective\) is zero when no wind is apparent as there is no natural convection due to the forced air flow through the absorber comb.
As the convection under the influence of wind disturbance is highly complex and subject to current research, convective losses are not considered
\begin{equation}
    \QDotLossAbsorberCupConvective = 0
\end{equation}

The loss due to the radiation of the hot absorber cup can be described by the Stefan-Boltzmann law
\begin{equation}
    \QDotLossAbsorberCupRadiative = \epsilon \sigma \SurfaceFront (\TAbsorber^4-\TAmbient^4)
\end{equation}
with the Stefan-Boltzmann constant \(\sigma\) and the ambient temperature \(\TAmbient\).
As the absorber absorbs the solar irradiance at different wave lengths than it emits the thermal radiation, one could use different a different absorption \(\alpha_{\mathrm{sol}}\) and emission \(\epsilon\) coefficient.
However, as the model is kept simple, both coefficients are set equally.

Finally, \(\QDotAbsorberComb\) can be written as the differences of the enthalpy of the incoming air with \(\TInlet{1}\) and the air leaving the comb with \(\TInlet{2}\).
\begin{align}
    \QDotAbsorberComb & = \EnthalpyFlowTInlet{2} - \EnthalpyFlowTInlet{1}                                   \\
& = \MDotAbsorberCup \cdot ( \SpecificEnthalpyTInlet{2} - \SpecificEnthalpyTInlet{1})\\
& = \HeatTransferCoefficientAbsorber \HeatTransferSurface (\TAbsorber-T_{\mathrm{m}})
\end{align}
where \(\HeatTransferCoefficientAbsorber\) is the heat transfer coefficient and \(\HeatTransferSurface\) is the contact area between the absorber comb and air.
Strictly speaking, in equation above, the mean logarithmic temperature difference should be used to calculate the heat flux.
However, this will be omitted in a first approximation.
Instead, a mean temperature is calculated from the incoming \(\TInlet{1}\) and outgoing  \(\TInlet{2}\) air temperature.
\begin{equation}
    T_{\mathrm{m}} = (1-\TemperatureWeightingFactor) \TInlet{1} + \TemperatureWeightingFactor \TInlet{2}
\end{equation}
\(\TemperatureWeightingFactor\) represents an adjustable parameter.

Expanding \cref{eq:UDotAbsorber} by \cref{?} yields the following equation
\begin{equation}
    m_{\mathrm{abs}}  c_{\mathrm{abs}} \frac{d \TAbsorber}{d t} = \epsilon \SurfaceFront \FluxSolar -  \epsilon \sigma \SurfaceFront (\TAbsorber^4-\TAmbient^4) - \HeatTransferCoefficientAbsorber \HeatTransferSurface (\TAbsorber-T_{\mathrm{m}}))
    \label{eq:ODE}
\end{equation}

The temperature \(\TAbsorber\) of the absorber comb defines the only state of the model.
\(\FluxSolar\) is considered a manipulated variable.
Still unknown are the temperatures \(\TInlet{1}\) and \(\TInlet{2}\).
\(\TInlet{1}\) can be derived from an energy balance around the air in front of the comb (see \cref{?}).
\begin{equation}
\EnthalpyFlowTInlet{1} = arr \cdot \EnthalpyFlowTReturn{1} + (1-arr) \cdot \EnthalpyFlowTAmbient
\end{equation}
where \(arr\) is the air return ratio, which indicates the fraction of returned air that flows back into the absorber.
The equation can be transformed to
\begin{equation}
    \SpecificEnthalpyTInlet{1} = arr \cdot \SpecificEnthalpyTReturn{1} + (1-arr) \cdot \SpecificEnthalpyTAmbient
    \label{eq:SpecificEnthalpyTInlet1}
\end{equation}
One could also rewrite the specific enthalpy as the product of the heat capacity and the temperature \(h = c_p(T)T\).
In some applications \(c_p\) is considered constant, when the temperature does not change significantly.
This is not the case for the absorber cups, where the temperature may vary between \SIrange{100}{1200}{\degreeCelsius}.
\(c_p(T)\) may then be approximated by a high order polynomial.
However, to save computational effort the specific enthalpy is directly approximated by a low order polynomial as described in \cref{?}.
To compute \(T\) e.g., \(\TInlet{1}\), the inverse function \(T(h)\) is also approximated by a polynomial.
This is possible, because the approximation polynomial is strictly monotonically increasing in the relevant range and thus allows a closed-form solution for the inverse function.
Now we have a new unknown \(\SpecificEnthalpyTReturn{1}\), which can be calculated from energy balances around the transport zone.

\textbf{Transport zone}
The energy balance around the transport zone of the returned air (see \cref{?}) yields
\begin{equation}
\EnthalpyFlowTReturn{1} = \EnthalpyFlowTReturn{3} +\QDotLossAbsorberCupInternal
\end{equation}
which is equivalent to
\begin{equation}
\MDotAbsorberCup\SpecificEnthalpyTReturn{1} = \MDotAbsorberCup\SpecificEnthalpyTReturn{3}+\QDotLossAbsorberCupInternal
    \label{eq:SpecificEnthalpyTReturn1}
\end{equation}
Here, \(\QDotLossAbsorberCupInternal\) is the heat flow from the hot inlet air to the return air.
\(\QDotLossAbsorberCupInternal\) can be further divided into
\begin{equation}
    \QDotLossAbsorberCupInternal = \QDotLoss{comb} + \QDotLoss{tube,1} + \QDotLoss{tube,2}
\end{equation}
The heat losses in the pipe sections can be calculated using conventional formulas for heat transfer through a multi-layered pipe.
The absorber comb is converted to an equivalent tube, so that the same formula can be applied \cref{In Jan Gall}.
Using these simplifications results in
\begin{equation}
\QDotLossAbsorberCupInternal= \alpha_{\mathrm{i\to r}} \cdot A_{\mathrm{i\to r}} \cdot (\TInlet{2} - \TReturn{3})
    \label{eq:QDotLossAbsorberCupInternal}
\end{equation}
where \(\alpha_{\mathrm{i\to r}}\) is the heat
transfer coefficient. The geometry of the absorber, i.e. lengths and diameters, is shown in \cref{?}.
\begin{align}
\alpha_{\mathrm{i\to r}} & =\frac{\pi \cdot l_{\mathrm{B}}}{\frac{1}{\alpha_{\mathrm{inlet, 2}} \cdot d_{1, \mathrm{B}}}+\frac{1}{2} \cdot B+\frac{1}{\alpha_{\mathrm{return, 3}} \cdot d_{2, \mathrm{~B}}}} \\
& +\frac{\pi \cdot l_{\mathrm{C}}}{\frac{1}{\alpha_{\mathrm{inlet, 2}} \cdot d_{1, \mathrm{C}}}
        +\frac{1}{2} \cdot C+\frac{1}{\alpha_{\mathrm{return, 3}} \cdot d_{4, \mathrm{C}}}}
    \\
& +\frac{\pi \cdot l_{\mathrm{D}}}{\frac{1}{\alpha_{\mathrm{inlet, 2}} \cdot d_{1, \mathrm{D}}}+\frac{1}{2} \cdot D+\frac{1}{\alpha_{\mathrm{return, 3}} \cdot d_{3, \mathrm{D}}}},
    \label{eq:NTU}
\end{align}
with
\begin{align}
     & B=\frac{1}{\lambda_{\text {cer}}} \cdot \ln \frac{d_{2, \mathrm{B}}}{d_{1, \mathrm{B}}}                                                                                                                                                                              \\
     & C=\frac{1}{\lambda_{\text {ins}}} \cdot \ln \frac{d_{2, \mathrm{C}}}{d_{1, \mathrm{C}}}+\frac{1}{\lambda_{\text {cer}}} \cdot \ln \frac{d_{3, \mathrm{C}}}{d_{2, \mathrm{C}}}+\frac{1}{\lambda_{\text {pipe}}} \cdot \ln \frac{d_{4, \mathrm{C}}}{d_{3, \mathrm{C}}} \\
     & D=\frac{1}{\lambda_{\text {ins}}} \cdot \ln \frac{d_{2, \mathrm{D}}}{d_{1, \mathrm{D}}}+\frac{1}{\lambda_{\text {pipe}}} \cdot \ln \frac{d_{3, \mathrm{D}}}{d_{2, \mathrm{D}}}
    \label{eq:NTUConstants}
\end{align}
\Cref{eq:NTU} (respectively \cref{eq:NTUConstants}) depends on the geometry of the absorber, material properties (thermal conductivity \(\lambda\)) and the heat transfer coefficient \(\alpha\).
% NOTE: Shall we implement a temperature dependent thermal conductivity?
While the geometry is constant, the thermal conductivity and the heat transfer coefficients are temperature dependent.
The equations for these will be described in \cref{?}.

From \cref{eq:SpecificEnthalpyTInlet1, eq:SpecificEnthalpyTReturn1, eq:QDotLossAbsorberCupInternal} follows the missing equation for \(\SpecificEnthalpyTInlet{1}\)
\begin{equation}
\SpecificEnthalpyTInlet{1} = arr \cdot \frac{\alpha_{\mathrm{i\to r}} \cdot A_{\mathrm{i\to r}} \cdot (\TInlet{2} - \TReturn{3})}{\MDotAbsorberCup} + (1-arr) \cdot \SpecificEnthalpyTAmbient
    \label{eq:SpecificEnthalpyTInlet1Extended}
\end{equation}
The temperature of the return air \(\TReturn{3}\) is a measured variable in the power plant and therefore known.
Inserting \cref{eq:SpecificEnthalpyTInlet1Extended} in \cref{eq:ODE} yields an ordinary-differential equation with one unknown algebraic variable \(\TInlet{2}\).
The algebraic equation for \(\TInlet{2}\) can be derived from an energy balance around the air in the absorber comb
\begin{equation}
    0 = \MDotAbsorberCup (\SpecificEnthalpyTInlet{1}-\SpecificEnthalpyTInlet{2}) + \QDotAbsorberComb
    \label{eq:HBalanceHeatingInlet}
\end{equation}
which is equal to
\TODO{I think something is missing here.}
\begin{equation}
0 = \MDotAbsorberCup \left(arr \cdot \frac{\alpha_{\mathrm{i\to r}} \cdot A_{\mathrm{i\to r}} \cdot (\TInlet{2} - \TReturn{3})}{\MDotAbsorberCup} + (1-arr) \cdot \SpecificEnthalpyTAmbient-\SpecificEnthalpyTInlet{2}\right) + \HeatTransferCoefficientAbsorber \HeatTransferSurface (\TAbsorber-T_{\mathrm{m})}
\end{equation}
Thus, \cref{eq:SpecificEnthalpyTInlet1Extended, eq:ODE, eq:HBalanceHeatingInlet} describe an differential-algebraic equation system with one ordinary differential equation and one algebraic equation.
With \(\TAbsorber\) as the state and \(\TInlet{2}\) as an algebraic variable.

Finally, the outgoing air temperature \(\TInlet{3}\) can be calculated by an energy balance around the transport zone of the inlet air by
\begin{equation}
    \SpecificEnthalpyTInlet{3} = \SpecificEnthalpyTInlet{2} - \frac{\QDotLossAbsorberCupInternal}{\MDotAbsorberCup}
\end{equation}
The specfic enthalpy \(\SpecificEnthalpyTInlet{2}\) and the loss \(\QDotLossAbsorberCupInternal\) are known by solving the differential-algebraic equation as stated before.
The temperature \(\TInlet{3}\) can then be calculated from the inverse function \(T(h)\).

With all these equations, the whole absorber cup is defined.
To compute the output for an array of absorber cups each output can be calculated individually.
% NOTE: This sentence belongs to another section.
To increase the performance the output of the individual cups can be computed in parallel.

\textbf{Header}
The outflowing air of each cup is then mixed in the primary header.
The enthalpy of the mixture is given by
\begin{equation}
    \EnthalpyFlowTInlet{3,\mathrm{mixed}} = \sum_i \dot{m}_{\IndexAbsorber, i} \cdot \SpecificEnthalpyTInlet{3,i}
\end{equation}
The enthalpy flow of the outflowing air can be derived from an energy balance around the header, given by
\begin{equation}
    \EnthalpyFlowTInlet{4} = \EnthalpyFlowTInlet{3,\mathrm{mixed}} + \QDotLoss{header,1}
\end{equation}
with \(\QDotLoss{header,1}\) being the heat losses through the header walls.
The geometry of the headers is shown in \cref{?}.
To calculate the heat losses, the funnel-shaped elements are converted into an equivalent cylindrical shape and losses to the environment in the radial direction through the wall and insulation are taken into account.
This results in the following equation
\begin{equation}
\dot{Q}_{\text {loss}, \IndexHeader}=
\frac{\pi \cdot l_{\IndexHeader} \cdot\left(T_{\mathrm{inlet,3,mixed}}-T_{\text {amb}}\right)}
{\frac{1}{\alpha_{\IndexHeader} \cdot d_{1, \IndexHeader}}+\frac{1}{2} \cdot P+\frac{1}{\alpha_{\mathrm{amb}} \cdot d_{3, \mathrm{\IndexHeader}}}}
    \label{eq:QDotLossHeader1}
\end{equation}
with
\begin{equation}
    P=\frac{1}{\lambda_{\text {ins}}} \cdot \ln \frac{d_{2, \IndexHeader}}{d_{1, \IndexHeader}}+\frac{1}{\lambda_{\text {pipe}}} \cdot \ln \frac{d_{3, \IndexHeader}}{d_{2, \IndexHeader}}
\end{equation}
All parameters are constant header characteristics except for the heat transfer coefficient \(\alpha_{\IndexHeader}\) which is temperature dependent.
The equation fot the coefficient is derived in \cref{?}.
Analogous to \cref{QDotLossHeader1}, the equation of the secondary header can be set up
if the incoming air from the four primary headers is considered to be perfectly mixed.
Thus, it is finally possible to determine the temperature of the air leaving the receiver from an energy balance around the secondary header, given by
\begin{equation}
    \dot{H}_{\mathrm{out}} = \EnthalpyFlowTInlet{5} = \EnthalpyFlowTInlet{4,\mathrm{mixed}} + \QDotLoss{header,2}
\end{equation}
which is equal to
\begin{equation}
    h_{\mathrm{out}} = \SpecificEnthalpyTInlet{4,\mathrm{mixed}} + \frac{\QDotLoss{header,2}}{\MDotReceiver}
\end{equation}
where \(\MDotReceiver\) is the mass flow through the whole receiver, which is equal to the sum of the mass flows through each absorber cup.
\(\QDotLoss{header,2}\) corresponds to the heat loss in the secondary header.
Again \(T_{\mathrm{out}}\) can be calculated with the inverse function \(T(h)\).
The mass flow \(\MDotReceiver\) can be measured and is thus an input parameter of the model.

\textbf{Mass balance}
Since the receiver is modeled without pressure as a state, no losses are assumed in the mass balance, so that the mass flow of the recirculated air is equal to that of the heated air leaving the receiver.
Thus, the mass flow of each absorber cup \(\MDotAbsorberCup\) can be calculated from the receiver mass flow \(\MDotReceiver\).
However, as stated before each subreceiver has a adjustable valve, which may result in an different mass flow in each subreceiver.
Furthermore, each absorber cup has an orifice plate which adjusts the mass flow of the aborber cups in a way that it is optimized for the usual flux density distribution.
For further details we refer to \cref{?}.

Within this work, the valve for each subreceiver is considered open.
Thus, the mass flow distribution is only dependent of the orifice diameters.
The mass flow law for an orifice plate is given by \cref{?https://www.efunda.com/formulae/fluids/calc_orifice_flowmeter.cfm}
\begin{equation}
    \dot{V} = \alpha_0 A_0 \sqrt{\frac{2 \Delta p}{\rho}}
\end{equation}
which is equal to
\begin{equation}
    \dot{m} = \alpha_0 \frac{\pi \OrificeDiameter^2}{4} \sqrt{2 \rho_{\mathrm{air}} \Delta p }
\end{equation}
where \(\alpha_0\) is the flow coefficient, \(d_{}\) the orifice diameter,  \(\rho_{\mathrm{air}}\) the density of the air and \(\Delta p\) the pressure difference before and after the plate.
The flow coefficient is a function of the Reynolds number.
\TODO{What are the actual numbers?}
However, for high Reynolds numbers the coeeficient is almost constant, which is the case in our setup as we consider mass flows between \SIrange{2}{10}{\kilo\per\second} which yields Reynolds number between \numrange{100000000000}{1}.
For a constant mass flow, the pressure difference is constant as well.
\TODO{Is that correct? We are actually calculating \(\rho(T)\) when computing NTU or \(\QDotAbsorberComb\). What is the actual influence on the mass flow, when considering \(\rho\)? Shall we make a paramter study about that? Is that even possible?}
Furthermore, the air density is considered constant as no pressure change is considered in the model.
Thus the mass flow is directoly proportional to the square of the orifice diameter, i.e.
\begin{equation}
    \dot{m} \propto \OrificeDiameter^2
\end{equation}
Hence, the mass flow of one absorber cup can be calculated by
\begin{equation}
    \dot{m}_{\IndexAbsorber, i} =\MDotReceiver \cdot \frac{\OrificeDiameter[,i]^2}{\sum_j^{n_{\mathrm{cups}}}\OrificeDiameter[,j]^2}
\end{equation}
\begin{equation}
    The
\end{equation}
END.
% NOTE: Shall we implement that as well?
% For the temperature Tinlet,2 an approximation is given as in equation (6.9).
% by the absorber temperature TAbs is used.
% m_{\mathrm{Abs}} \cdot c_{\mathrm{Abs}} \cdot \frac{d T_{\mathrm{Abs}}}{d t} & =\eta_{\mathrm{Abs}} \cdot P_{\mathrm{sol}}-\QDotAbsorberComb


% Describe how these can be reformulated to get one ode and one algebraic equation.

% Describe the approximation for the specific enthalpy.

% Describe the approximation for the ntu coefficient.

% Describe the primary and secondary header balances and express them as explicit functions.

% Describe how the mass flow is calculated by the orifice diameters.

\section{Absorber Cup Extension}

In order to better resemble the behavior of the front and outlet temperature the model is extended by an extra state.
This is done by dividing the absorber comb into a front and a back part.
It has the effect that it dampens the increase of the \(\TInlet{3}\) output temperature of the absorber.
Thus, the front temperature can rise faster than the \(\TInlet{3}\) output temperature.
Due to the second state the behavior of the DAE system can be compared to a second order controlled plant (PT2).
Beacuse of the additional state also heat conduction \(\QDotConduction\) has to be considered.
Thus, \cref{eq:UDotAbsorber} changes to
\begin{equation}
\UDotAbsorberFront =\QDotSol - \QDotLossAbsorberCupConvective -  \QDotLossAbsorberCupRadiative -\QDotAbsorberCombFront - \QDotConduction
    \label{eq:UDotAbsorberExtended}
\end{equation}
with \(\QDotConduction\) being the heat conduction from the absorber front to the absorber back.
\(\QDotAbsorberCombFront\) represents the heat convection from the absorber comb to the air and can be described by
\begin{align}
\QDotAbsorberCombFront & = \EnthalpyFlowTInlet{\mathrm{1b}} - \EnthalpyFlowTInlet{1}                                             \\
& = \MDotAbsorberCup \cdot ( \SpecificEnthalpyTInlet{\mathrm{1b}} - \SpecificEnthalpyTInlet{1})           \\
& = \HeatTransferCoefficientAbsorberFront \HeatTransferSurfaceFront (\TAbsorberFront-T_{\mathrm{m,front}})
\end{align}
where \(T_{\mathrm{m,front}}\) is defined as
\begin{equation}
T_{\mathrm{m,front}} = (1-\TemperatureWeightingFactorFront) \TInlet{1} + \TemperatureWeightingFactorFront \TInlet{\mathrm{1b}}
\end{equation}
\begin{equation}
    \QDotConduction = \lambda_{\mathrm{comb}} \cdot \frac{\SurfaceFrontSolid \cdot (\TAbsorberFront-\TAbsorberBack)}{l_{\mathrm{comb}/2}}
\end{equation}
where \(\SurfaceFrontSolid\) is the surface area \(\SurfaceFront\) of the absorber cup minus the area of the channels where the air flows through.
\(l_{\mathrm{comb}}\) is the depth of the absorber comb.
Due to the volumetric effect of the porous absorber comb not all of the radiation is absorbed on the surface front but also some radiation enters in the comb and is absorbed by the back part of the comb.
This characteristic is considered by the value \(\xi_{\mathrm{rad}}\) which splits up the solar power into a front and back part i.e.
\begin{align}
    \PowerSolarFront & = \xi_{\mathrm{rad}} \PowerSolar     \\
    \PowerSolarBack  & = (1-\xi_{\mathrm{rad}}) \PowerSolar
\end{align}
Combining all the equations above yields
\begin{equation}
m_{\mathrm{abs}}  c_{\mathrm{abs}} \frac{d \TAbsorber}{d t} = \epsilon \xi_{\mathrm{rad}} \SurfaceFront \FluxSolar -  \epsilon \sigma \SurfaceFront (\TAbsorberFront^4-\TAmbient^4) - \HeatTransferCoefficientAbsorberFront \HeatTransferSurfaceFront (\TAbsorberFront-T_{\mathrm{m,front}}) - \lambda_{\mathrm{comb}} \cdot \frac{\SurfaceFrontSolid \cdot (\TAbsorberFront-\TAbsorberBack)}{l_{\mathrm{comb}/2}}
\end{equation}
The equation for the absorber back is similar, however for the absorber back no heat dissipation is considered, which results in the following equation
\begin{equation}
\UDotAbsorberBack =\QDotSol -\QDotAbsorberCombBack + \QDotConduction
\end{equation}
where \(\QDotAbsorberCombBack\) is formulated by
\begin{align}
\QDotAbsorberCombBack & = \EnthalpyFlowTInlet{2} - \EnthalpyFlowTInlet{\mathrm{1b}}                                           \\
& = \MDotAbsorberCup \cdot ( \SpecificEnthalpyTInlet{2} - \SpecificEnthalpyTInlet{\mathrm{1b}})         \\
& = \HeatTransferCoefficientAbsorberBack \HeatTransferSurfaceBack (\TAbsorberBack-T_{\mathrm{m,back}})
\end{align}
\begin{equation}
T_{\mathrm{m,back}}  = (1-\TemperatureWeightingFactorBack) \TInlet{\mathrm{1b}} + \TemperatureWeightingFactorBack \TInlet{2}
\end{equation}
