%
% ============ !!! VOR BEGINN AUSFÜLLEN !!! ===========
% =====================================================

\newcommand{\workTyp}{Bachelorarbeit\xspace}										% <Typ> der Arbeit, z.B. Bachelorarbeit oder Projektarbeit
\newcommand{\workTitel}{Modellprädiktive Regelung eines keramischen Receivers für Solartürme\xspace}								% <Titel> der Arbeit
\newcommand{\workAutor}{Markus Tobias Geschonneck\xspace}					% <Name> des Autors
\newcommand{\workMatrikelnummer}{11131469\xspace}	% <Matrikelnummer> des Studierenden
\newcommand{\workAbgabe}{06.06.2023\xspace}	%\today für aktuelles Datum			% <Datum> der Abgabe (TT.MM.JJJ)
\newcommand{\workAusgabe}{04.04.2023\xspace}				% <Datum> der Ausgabe es Themas (TT.MM.JJJ)
\newcommand{\workReferent}{Dr. Chong Dae Kim\xspace}		% <Referent> der Arbeit, z.B. Prof. Dr.-Ing. Mohieddine Jelali
\newcommand{\workKorreferent}{M. Sc. David Zanger\xspace}% <Korreferent> der Arbeit 
\newcommand{\workStadtDatum}{Leverkusen, den 06.06.2023\xspace}					% <Stadt> der Institution

% ================== ALLGEIME ANGABEN =================
% =====================================================
\newcommand{\workTodo}[1]{\textcolor{black}{#1}}						% ToDo kennzeichnen

\newcommand{\workFakultaet}{Fakultät für Anlagen, Energie- und Maschinensysteme\xspace} % <Fakultaet> 
\newcommand{\workLab}{<Labor>\xspace}	% <Institution> 
\newcommand{\workInstitution}{Technische Hochschule Köln\xspace}	% <Institution> 
\newcommand{\workCompany}{Deutsches Zentrum für Luft- und Raumfahrt\xspace}
\newcommand{\workInstitut}{Institut für Solarforschung\xspace}	% <Institut> 
\newcommand{\workProf}{<Betreuender Professor>\xspace}		% <Name> des Professors

% =============== MACROS AND NEW COMMANDS ===============
% =======================================================

% =============== NEW COMMANDS ===============
% ============================================
\newcommand{\gans}[1]{\glqq #1\grqq}					% Anführungszeichen unten...oben
\newcommand{\quot}[1]{\grqq #1\grqq}					% quotationmarks up..up

% =============== HYPENATION ===============
% ==========================================
\hyphenation{Aus-ga-be-for-mat Ein-heits-sprung-ant-wort}	% {word-one word-two ...}

% ================= FARBEN =================
% ==========================================
\definecolor{violett}{RGB}{14 7 110}
\definecolor{gruen}{RGB}{0 150 0}

% ================= EXTRAS =================
% ==========================================
\newcommand\myemptypage{
    \null
    \thispagestyle{empty}
    \addtocounter{page}{-1}
    \newpage
    }

% define custom page style for empty pages
\fancypagestyle{empty_with_pagenumber}{
    \fancyhf{} % clear header and footer
    \fancyfoot[RO,LE]{\thepage} % set page number to outer position
    \renewcommand{\headrulewidth}{0pt} % remove header rule
    \renewcommand{\footrulewidth}{0pt} % remove footer rule
}
\newcommand\emptywithpagenumber{
    \null
    \thispagestyle{empty_with_pagenumber}
    \newpage
}
