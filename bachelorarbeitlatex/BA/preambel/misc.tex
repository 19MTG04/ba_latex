%
% ============ !!! VOR BEGINN AUSFÜLLEN !!! ===========
% =====================================================

\newcommand{\workTyp}{Bachelorarbeit\xspace}										% <Typ> der Arbeit, z.B. Bachelorarbeit oder Projektarbeit
\newcommand{\workTitel}{<Titel der Arbeit>\xspace}								% <Titel> der Arbeit
\newcommand{\workAutor}{Felix Maas\xspace}					% <Name> des Autors
\newcommand{\workMatrikelnummer}{11131764\xspace}	% <Matrikelnummer> des Studierenden
\newcommand{\workAbgabe}{<Datum der Abgabe>\xspace}	%\today für aktuelles Datum			% <Datum> der Abgabe (TT.MM.JJJ)
\newcommand{\workAusgabe}{<Datum der Ausgabe des Themas>\xspace}				% <Datum> der Ausgabe es Themas (TT.MM.JJJ)
\newcommand{\workReferent}{<Referent>\xspace}		% <Referent> der Arbeit, z.B. Prof. Dr.-Ing. Mohieddine Jelali
\newcommand{\workKorreferent}{<Koreferent>\xspace}% <Korreferent> der Arbeit 
\newcommand{\workStadtDatum}{Köln, den xxx\xspace}					% <Stadt> der Institution
\newcommand{\workKolloqium}{<Datum des Kolloqiums>\xspace}

% ================== ALLGEIME ANGABEN =================
% =====================================================
\newcommand{\workTodo}[1]{\textcolor{black}{#1}}						% ToDo kennzeichnen

\newcommand{\workFakultaet}{Fakultät für Anlagen, Energie- und Maschinensysteme\xspace} % <Fakultaet> 
\newcommand{\workLab}{<Labor>\xspace}	% <Institution> 
\newcommand{\workInstitution}{Technische Hochschule Köln\xspace}	% <Institution> 
\newcommand{\workCompany}{Deutsches Zentrum für Luft- und Raumfahrt\xspace}
\newcommand{\workInstitut}{Institut für Solarforschung\xspace}	% <Institut> 
\newcommand{\workProf}{<Betreuender Professor>\xspace}		% <Name> des Professors

% =============== MACROS AND NEW COMMANDS ===============
% =======================================================

% =============== NEW COMMANDS ===============
% ============================================
\newcommand{\gans}[1]{\glqq #1\grqq}					% Anführungszeichen unten...oben
\newcommand{\quot}[1]{\grqq #1\grqq}					% quotationmarks up..up

% =============== HYPENATION ===============
% ==========================================
\hyphenation{Aus-ga-be-for-mat Ein-heits-sprung-ant-wort}	% {word-one word-two ...}

% ================= FARBEN =================
% ==========================================
\definecolor{violett}{RGB}{14 7 110}
\definecolor{gruen}{RGB}{0 150 0}

% ================= EXTRAS =================
% ==========================================
\newcommand\myemptypage{
    \null
    \thispagestyle{empty}
    \addtocounter{page}{-1}
    \newpage
    }