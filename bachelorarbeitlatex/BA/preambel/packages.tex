% ========================================================================
\usepackage[utf8]{inputenc}							% !UTF-8
%
% ~~~~~~~~~~~~~~~~~~~~~~~~~~~~~~~~~~~~~~~~~~~~~~~~~~~~~~~~~~~~~~~~~~~~~~~~
% 1. Einige Pakete müssen vor allen anderen geladen werden
% ~~~~~~~~~~~~~~~~~~~~~~~~~~~~~~~~~~~~~~~~~~~~~~~~~~~~~~~~~~~~~~~~~~~~~~~~
\usepackage{xspace} 								% Define commands that don't eat spaces.
\usepackage[ngerman]{babel}					% Languagesetting: Sprachpaket wird in \documentclass für gesamtes Dokument festgelegt

\usepackage[hyperref]{xcolor} 			% benutzerdefinierte Farben
\usepackage[]{graphicx} 	 					% für Bilder
%\usepackage[fleqn]{amsmath}					% Amsmath - Mathematik Basispaket fl=flush left = linksbündig
\usepackage{amsmath}                            % Amsmath - Mathematik Basispaket zentriert
\usepackage{amssymb}
\usepackage{tikz}		    						% für Graphiken und Zeichnen
\usepackage{pgfplots}								% Für MATLAB2Latex export
\usepackage{varwidth}



% ~~~~~~~~~~~~~~~~~~~~~~~~~~~~~~~~~~~~~~~~~~~~~~~~~~~~~~~~~~~~~~~~~~~~~~~~
% Fonts und Seitenränder
% ~~~~~~~~~~~~~~~~~~~~~~~~~~~~~~~~~~~~~~~~~~~~~~~~~~~~~~~~~~~~~~~~~~~~~~~~
\usepackage[T1, EU1]{fontenc} 							% T1 Schrift Encoding (notwendig für die meisten Type 1 Schriften)
\usepackage{textcomp}
\usepackage{lmodern}


\usepackage[            
	left=3.5cm,										% linke Randbreite                                                                     
	right=2cm,										% rechte Randbreite                                                                  
	top=2.7cm,										% oberer Rand                                                                         
	bottom=3cm									% unterer Rand
]{geometry}											% Seitenlayout verändern                    

%%%% Head and Footline
\usepackage{fancyhdr} % for page style customization

% define a new page style for chapter pages
\fancypagestyle{chapterstart}{%
    \fancyhf{}% clear header and footer
    \renewcommand{\headrulewidth}{0pt}% remove horizontal line
}

% set the chapter start page style as 'plain' for all chapter pages
\let\origchapter\chapter
\renewcommand\chapter{\clearpage\thispagestyle{chapterstart}\origchapter}

% set the fancyhdr page style for the rest of the document
\pagestyle{fancy}
\fancyhf{} % clear header and footer
\fancyfoot[RO,LE]{\thepage} % set page number to outer position
\fancyhead[RO]{\sffamily\nouppercase{\rightmark}}
\fancyhead[LE]{\sffamily\nouppercase{\leftmark}}
\renewcommand{\chaptermark}[1]{\markboth{\thechapter.\ #1}{}}
\renewcommand{\sectionmark}[1]{\markright{\thesection.\ #1}}

% Stop Latex from stretching vertically, which is activated for twosided documents automatically
\raggedbottom


%% Inhaltsverzeichnis

\usepackage{tocloft}

\newcommand{\subsectionfont}{\normalfont}
\newcommand{\numberfont}{\normalfont}

\renewcommand{\cftchapfont}{\subsectionfont}
\renewcommand{\cftsecfont}{\subsectionfont}
\renewcommand{\cftsubsecfont}{\subsectionfont}

\renewcommand{\cftchappagefont}{\numberfont}
\renewcommand{\cftsecpagefont}{\numberfont}
\renewcommand{\cftsubsecpagefont}{\numberfont}

\renewcommand{\cftpartleader}{\cftdotfill{\cftdotsep}} % for parts
\renewcommand{\cftchapleader}{\cftdotfill{\cftdotsep}} % for chapters

\setlength{\cftbeforechapskip}{3pt}

%\usepackage{tocstyle}  Die folgenden 3 Zeilen mussten auskommentiert werden. Welche Auswirkungen das hat, weiß ich nicht.
%\settocfeature[toc][0]{entryhook}{}                                                                
%\settocfeature[0]{entryvskip}{0.5em plus 1pt}
% ~~~~~~~~~~~~~~~~~~~~~~~~~~~~~~~~~~~~~~~~~~~~~~~~~~~~~~~~~~~~~~~~~~~~~~~~                                  
% 3. Text related packages                                                                                  
% ~~~~~~~~~~~~~~~~~~~~~~~~~~~~~~~~~~~~~~~~~~~~~~~~~~~~~~~~~~~~~~~~~~~~~~~~                                  
\usepackage[hyphens]{url} 							% Setzen von URLs. In Verbindung mit hyperref sind diese auch aktive Links.
                                                                                                         
% ~~~~~~~~~~~~~~~~~~~~~~~~~~~~~~~~~~~~~~~~~~~~~~~~~~~~~~~~~~~~~~~~~~~~~~~~                                  
% 4. PDF related packages                                                                                   
% ~~~~~~~~~~~~~~~~~~~~~~~~~~~~~~~~~~~~~~~~~~~~~~~~~~~~~~~~~~~~~~~~~~~~~~~~                                  
\usepackage[                                                                                      
	colorlinks=true,	        						% Links erhalten Farben statt Kaestchen                                      
	urlcolor=black,
	filecolor=black, 
	linkcolor=black, 
	citecolor=black,                                                               
]{hyperref}											% Aktivierung für Referenzen zur Erstellung der pdf                                   
                                                   
% ~~~~~~~~~~~~~~~~~~~~~~~~~~~~~~~~~~~~~~~~~~~~~~~~~~~~~~~~~~~~~~~~~~~~~~~~                                  
% 5. Tables (Tabular)
\usepackage{colortbl}
% ~~~~~~~~~~~~~~~~~~~~~~~~~~~~~~~~~~~~~~~~~~~~~~~~~~~~~~~~~~~~~~~~~~~~~~~~
\usepackage{booktabs}								% horizontale Linien in Tabellen
\usepackage{multirow}								% Um in Tabellen über mehrere Zeilen gleichzeitig schreiben zu können
\usepackage{longtable}              % Tabellen über mehrere Seiten          
                                                                                                           
% ~~~~~~~~~~~~~~~~~~~~~~~~~~~~~~~~~~~~~~~~~~~~~~~~~~~~~~~~~~~~~~~~~~~~~~~~                                  
% 7. Figures and placement                                                                                  
% ~~~~~~~~~~~~~~~~~~~~~~~~~~~~~~~~~~~~~~~~~~~~~~~~~~~~~~~~~~~~~~~~~~~~~~~~                                  
\graphicspath{{fig/}}								% Pfad für Bilder. Kann im folgenden Dokument dann weggelassen werden          
\usepackage{float}									% Stellt die Option [H] für Floats zur Verfgung:\begin{figure}[H]
                                                                                                   
%========================== Zeilenabstand ================================                                  
\usepackage{setspace}								% Zeilenabstand festlegen; generell oder im Fließtext mit Befehl               
%\doublespacing	        							% 2-facher Abstand                                                             
%\onehalfspacing  									% 1,5-facher Abstand                                                              
%\singlespacing										% 1-facher Abstand                                                                 
                                                                                                         
%==================== Captions (Schrift, Aussehen) =======================
\usepackage{caption}								% Aussehen der Captions
\captionsetup{
	margin = 10pt,
	font = {small,rm},
	labelfont = {small,bf, sf},
	format = plain, 								% oder 'hang'
	indention = 0em,	 							% Einruecken der Beschriftung
	labelsep = colon, 								%period, space, quad, newline
	justification = RaggedRight, 					% justified, centering
	singlelinecheck = true, 						% false (true=bei einer Zeile immer zentrieren)
	position = bottom 								% top
}
%%% Bugfix Workaround
\DeclareCaptionOption{parskip}[]{}
\DeclareCaptionOption{parindent}[]{}
% Aussehen der Captions fuer subfigures (subfig-Paket)
\captionsetup[subfloat]{
	margin = 10pt,
	font = {small,sf},
	labelfont = {small,bf},
	format = plain, 								% oder 'hang'
	indention = 0em,	 							% Einruecken der Beschriftung
	labelsep = space, 								%period, space, quad, newline
	justification = RaggedRight, 					% justified, centering
	singlelinecheck = true, 						% false (true=bei einer Zeile immer zentrieren)
	position = bottom, 								% top
	labelformat = parens 							% simple, empty % Wie die Bezeichnung gesetzt wird
}
%\renewcaptionname{ngerman}{\contentsname}{Inhalt}
\renewcaptionname{ngerman}{\listfigurename}{Abbildungen}
%\renewcaptionname{ngerman}{\listtablename}{Tabellen}
%\renewcaptionname{ngerman}{\figurename}{Bild}
\renewcaptionname{ngerman}{\tablename}{Tabelle}
%
%======== Fussnoten =============================================================
\usepackage{footnote}								% fußnoten in beschriftung am fuß der ganzen seite

% ~~~~~~~~~~~~~~~~~~~~~~~~~~~~~~~~~~~~~~~~~~~~~~~~~~~~~~~~~~~~~~~~~~~~~~~~
% 9. WEITERE / EXTRAS
% ~~~~~~~~~~~~~~~~~~~~~~~~~~~~~~~~~~~~~~~~~~~~~~~~~~~~~~~~~~~~~~~~~~~~~~~~

%Abkürzungsverzeichnis-Packages
\usepackage{acronym}
\usepackage{siunitx}
\usepackage{amsmath}
\usepackage{tocloft}

%Excel2Latex
\usepackage{multicol}
\usepackage{multirow}
\usepackage{tabularx}
\usepackage{xcolor}
\usepackage{booktabs}

%Leere Seiten
\usepackage{afterpage}